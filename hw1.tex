% Copyright (C)  2015  Chen Ruichao <linuxer.sheep.0x@gmail.com>.
% This work is licensed under the Creative Commons Attribution-ShareAlike 4.0 International License.
% To view a copy of this license, visit http://creativecommons.org/licenses/by-sa/4.0/.

\documentclass{article}

% useful packages
\usepackage[colorlinks=true]{hyperref}  % clickable links
\usepackage{enumerate}                  % customize enumerated lists
\usepackage{tabu}                       % table on steroids
\usepackage{multirow}                   % for table spanning
\usepackage{amsmath}
\usepackage{amssymb}                    % for \nmid
\usepackage{amsfonts}                   % for \mathbb

% CS 70-specific
\renewcommand{\implies}{\Rightarrow}    % AMS uses logical implication
\newcommand{\Implies}{\Longrightarrow}  % CS 70 uses material implication
                                        % TODO use AMS \implies for our \Implies
\newcommand{\bicond}{\Leftrightarrow}   % \equiv for logical equivalence
                                        % \bicond for material equivalence

% packages for Creative Commons
\usepackage{graphicx}
\usepackage[rightcaption]{sidecap}
\usepackage{caption}
\usepackage{xmpincl}
\includexmp{license/cc_by-sa}

\begin{document}

\pagenumbering{gobble}
\setlength{\parindent}{0px}





\title{CS 70 Fall 2015 hw 1 solution}
\author{Chen Ruichao \texttt{<linuxer.sheep.0x@gmail.com>}}
\date{} % no date
\maketitle





\section{Truth table}
\begin{enumerate}[(a)]

    \item % equivalent or not?

          \begin{tabu}{cc|cc}
              $  $&$  $&$  $&$  $ \\
              \hline
               &  &  &  \\
               &  &  &  \\
               &  &  &  \\
               &  &  &  \\
          \end{tabu}

    \item % equivalent or not?

          \begin{tabu}{ccc|cc}
              $  $&$  $&$  $&$  $&$  $ \\
              \hline
                &   &   &   &   \\
                &   &   &   &   \\
                &   &   &   &   \\
                &   &   &   &   \\
                &   &   &   &   \\
                &   &   &   &   \\
                &   &   &   &   \\
                &   &   &   &   \\
          \end{tabu}

    \item % equivalent or not?

          \begin{tabu}{ccc|cc}
              $  $&$  $&$  $&$  $&$  $ \\
              \hline
                &   &   &   &   \\
                &   &   &   &   \\
                &   &   &   &   \\
                &   &   &   &   \\
                &   &   &   &   \\
                &   &   &   &   \\
                &   &   &   &   \\
                &   &   &   &   \\
          \end{tabu}

    \item % equivalent or not?

          \begin{tabu}{cc|cc}
              $  $&$  $&$  $&$  $ \\
              \hline
               &  &  &  \\
               &  &  &  \\
               &  &  &  \\
               &  &  &  \\
          \end{tabu}

\end{enumerate}





\section{Proposition}
\begin{enumerate}[(a)]
    \item
    \item
    \item
\end{enumerate}





\section{Predicate}
\begin{enumerate}[(a)]
    \item
    \item
    \item
\end{enumerate}





\section{Party}
\newcommand{\partyrule}[2]{\forall x \; \big(#1 \implies #2\big)}
Note: $P(x)$ should actually be ``$x$ is going to a party'', not ``$x$ goes to a party''.

\begin{enumerate}[(a)]
    \item \begin{enumerate}[(I)]
              \item $\partyrule{REPLACE ME with antecedent}{REPLACE ME with consequence}$
              \item $\partyrule{REPLACE ME}{REPLACE ME}$
              \item $\partyrule{REPLACE ME}{REPLACE ME}$
              \item $\partyrule{REPLACE ME}{REPLACE ME}$
              \item $\partyrule{REPLACE ME}{REPLACE ME}$
              \item $\partyrule{REPLACE ME}{REPLACE ME}$
          \end{enumerate}

    \item \begin{enumerate}[(I)]
              \item $\partyrule{REPLACE ME}{REPLACE ME}$
              \item $\partyrule{REPLACE ME}{REPLACE ME}$
              \item $\partyrule{REPLACE ME}{REPLACE ME}$
              \item $\partyrule{REPLACE ME}{REPLACE ME}$
              \item $\partyrule{REPLACE ME}{REPLACE ME}$
              \item $\partyrule{REPLACE ME}{REPLACE ME}$
          \end{enumerate}

    \item

\end{enumerate}





\section{Karnaugh Maps}
\begin{enumerate}[(a)]

    \item \begin{tabu}{cccc|c}
              $ A $&$ B $&$ C $&$ D $&$ Z $ \\
              \hline
              0 & 0 & 0 & 0 &  \\ % ← FILL IN HERE
              0 & 0 & 0 & 1 &  \\ % ← FILL IN HERE
              0 & 0 & 1 & 0 &  \\ % ← FILL IN HERE
              0 & 0 & 1 & 1 &  \\ % ← FILL IN HERE
              0 & 1 & 0 & 0 &  \\ % ← FILL IN HERE
              0 & 1 & 0 & 1 &  \\ % ← FILL IN HERE
              0 & 1 & 1 & 0 &  \\ % ← FILL IN HERE
              0 & 1 & 1 & 1 &  \\ % ← FILL IN HERE
              1 & 0 & 0 & 0 &  \\ % ← FILL IN HERE
              1 & 0 & 0 & 1 &  \\ % ← FILL IN HERE
              1 & 0 & 1 & 0 &  \\ % ← FILL IN HERE
              1 & 0 & 1 & 1 &  \\ % ← FILL IN HERE
              1 & 1 & 0 & 0 &  \\ % ← FILL IN HERE
              1 & 1 & 0 & 1 &  \\ % ← FILL IN HERE
              1 & 1 & 1 & 0 &  \\ % ← FILL IN HERE
              1 & 1 & 1 & 1 &  \\ % ← FILL IN HERE
          \end{tabu}

    \item \begin{tabu}{cc|cccc}
              \multicolumn{2}{c}{} & \multicolumn{4}{c}{$CD$} \\
              \multicolumn{2}{c}{} & 00 & 01 & 11 & 10 \\ \cline{3-6}
              \multirow{4}{*}{$AB$}
                  & 00 &   &   &   &   \\ % ← TO BE FILLED IN
                  & 01 &   &   &   &   \\ % ← TO BE FILLED IN
                  & 11 &   &   &   &   \\ % ← TO BE FILLED IN
                  & 10 &   &   &   &   \\ % ← TO BE FILLED IN
          \end{tabu}

    \item

    \item

\end{enumerate}





\section{Contrapositive and converse}
\begin{enumerate}[(a)]
    \item
    \item The contrapositive is REPLACE ME. % true or false?
    \item The converse is REPLACE ME. % true or false?
\end{enumerate}





\section{Proof practice}
\begin{enumerate}[(a)]
    \item
    \item
    \item
    \item
    \item
    \item
\end{enumerate}





\section{Gate keepers}





\newpage
Copyright 2015 \myname{} \texttt{<\myemail>}
\begin{SCfigure}[2.0]
    \centering
    \input{license/cc_by-sa.pdf_tex}
    \caption*{This work by \myname{} is licensed under a \href{http://creativecommons.org/licenses/by-sa/4.0/}{Creative Commons Attribution-ShareAlike 4.0 International License}.}
\end{SCfigure}

\end{document}
